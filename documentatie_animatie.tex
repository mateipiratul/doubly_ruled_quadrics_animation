\documentclass[a4paper,11pt]{article}
\usepackage[a4paper,margin=0.5in]{geometry}
\usepackage{tcolorbox}
\usepackage{listings}
\usepackage{xcolor}
\usepackage{amsmath}
\usepackage{graphicx}
\usepackage{float}

\definecolor{codegray}{rgb}{0.9,0.9,0.9}
\definecolor{darkblue}{rgb}{0,0,0.5}
\definecolor{listingbg}{rgb}{0.95,0.95,0.95}

\lstdefinestyle{mystyle}{
    backgroundcolor=\color{listingbg},   
    commentstyle=\itshape\color{darkblue},
    keywordstyle=\bfseries\color{blue},
    numberstyle=\tiny\color{gray},
    stringstyle=\color{red!80!black},
    basicstyle=\ttfamily\footnotesize,
    breaklines=true,
    numbers=left,                    
    numbersep=5pt,                  
    showspaces=false,                
    showstringspaces=false,
    showtabs=false,                  
    tabsize=4,
    frame=tb, % Adaugă un cadru sus și jos la listing
    captionpos=b % Poziția caption-ului sub listing
}
\lstset{style=mystyle} % Setează stilul global

% O cutie personalizată pentru definiții sau secțiuni importante
\newtcolorbox{definitionbox}[1]{
    colback=blue!5!white, % Culoare de fundal
    colframe=blue!60!black, % Culoare cadru
    fonttitle=\bfseries,
    title=#1,
    sharp corners,
    boxrule=1pt % Grosimea liniei cadrului
}

\title{\Huge Animația unei Cuadrice Dublu Riglate}
\author{Cocu Matei-Iulian și Lăutaru Bianca-Maria}
\date{\today} % Adaugă data curentă

\begin{document}

\maketitle
\begin{center}
    \textit{Documentație pentru generarea și vizualizarea unui hiperboloid cu o pânză}
\end{center}
\vspace{1cm}

\tableofcontents
\newpage

\section{Introducere}
Acest document descrie procesul de generare a unei animații tridimensionale pentru o cuadrică dublu riglată, specific un hiperboloid cu o pânză, utilizând limbajul de programare Python și biblioteca Matplotlib. Scopul este de a vizualiza cele două familii de drepte (generatoare) care compun suprafața cuadricei și de a observa structura acesteia în timpul rotației. Codul Python implementează calculul parametric al acestor drepte și orchestrarea animației.

\section{Cuadrice Dublu Riglate - Aspecte Teoretice}

\begin{definitionbox}{Definiție}
O \textbf{suprafață riglată} este o suprafață generată prin mișcarea unei drepte în spațiu. Această dreaptă mobilă se numește \textbf{generatoare} a suprafeței.
O \textbf{cuadrică dublu riglată} este o suprafață cuadrică (definită de o ecuație de gradul al doilea) care conține nu una, ci \textit{două} familii distincte de generatoare. Prin fiecare punct al unei astfel de suprafețe trec exact două drepte distincte care sunt în întregime conținute în suprafață, câte una din fiecare familie.
\end{definitionbox}

Principalele tipuri de cuadrice dublu riglate sunt:
\begin{itemize}
    \item \textbf{Hiperboloidul cu o pânză:} Este suprafața cuadrică definită canonic de ecuația:
    \[ \frac{x^2}{a^2} + \frac{y^2}{b^2} - \frac{z^2}{c^2} = 1 \]
    Aceasta este cuadrica implementată în codul Python furnizat.
    \item \textbf{Paraboloidul hiperbolic:} Definit canonic de ecuația:
    \[ \frac{x^2}{a^2} - \frac{y^2}{b^2} = 2z \]
\end{itemize}

\subsection{Generatoarele Hiperboloidului cu o Pânză}
Pentru hiperboloidul cu o pânză $\frac{x^2}{a^2} + \frac{y^2}{b^2} - \frac{z^2}{c^2} = 1$, cele două familii de generatoare pot fi descrise parametric. Un sistem de ecuații parametrice pentru aceste drepte este:

\textbf{Familia 1 ($\lambda$-const):}
\[
\begin{cases}
\frac{x}{a} - \frac{z}{c} = \lambda \left(1 - \frac{y}{b}\right) \\
\frac{x}{a} + \frac{z}{c} = \frac{1}{\lambda} \left(1 + \frac{y}{b}\right)
\end{cases}
\]
unde $\lambda$ este un parametru real.

O parametrizare mai directă, utilizată și în cod, pentru un punct $(x,y,z)$ de pe o generatoare în funcție de un parametru unghiular $\alpha$ (care identifică dreapta în familie) și un parametru $s$ (care definește poziția pe dreaptă, similar cu $z$) este:
\begin{itemize}
    \item \textbf{Familia 1 de generatoare:}
    \begin{align*}
    x(s, \alpha) &= a \left(\cos(\alpha) - \frac{s}{c_{eq}} \sin(\alpha)\right) \\
    y(s, \alpha) &= b \left(\sin(\alpha) + \frac{s}{c_{eq}} \cos(\alpha)\right) \\
    z(s, \alpha) &= s
    \end{align*}
    \item \textbf{Familia 2 de generatoare:}
    \begin{align*}
    x(s, \alpha) &= a \left(\cos(\alpha) + \frac{s}{c_{eq}} \sin(\alpha)\right) \\
    y(s, \alpha) &= b \left(\sin(\alpha) - \frac{s}{c_{eq}} \cos(\alpha)\right) \\
    z(s, \alpha) &= s
    \end{align*}
\end{itemize}
Aici, $c_{eq}$ corespunde parametrului $c$ din ecuația canonică a hiperboloidului. Aceste ecuații sunt implementate direct în scriptul Python pentru a genera punctele de pe fiecare linie.

\section{Descrierea Implementării Python}
Scriptul Python utilizează bibliotecile `numpy` pentru calcule numerice și `matplotlib` (cu `mplot3d` și `FuncAnimation`) pentru crearea graficelor 3D și a animației.

\subsection{Parametrii Scriptului}
Scriptul începe prin definirea unor parametri cheie:
\begin{itemize}
    \item \texttt{a, b}: Semiaxele elipsei de la $z=0$ a hiperboloidului.
    \item \texttt{c\_eq}: Parametru ce controlează curbura hiperboloidului și, implicit, înclinația generatoarelor. O valoare mai mică accentuează "gâtul" hiperboloidului.
    \item \texttt{z\_max}: Definește lungimea (sau intervalul pe axa $z$) pentru care sunt desenate segmentele de dreaptă.
    \item \texttt{num\_lines}: Numărul de drepte (generatoare) desenate pentru fiecare dintre cele două familii.
    \item \texttt{num\_points\_per\_line}: Numărul de puncte calculate pentru a desena fiecare segment de dreaptă, determinând rezoluția acestora.
    \item \texttt{num\_frames}: Numărul total de cadre (imagini individuale) din care va fi compusă animația finală.
    \item \texttt{fps}: Cadre pe secundă (frames per second), determinând viteza animației.
    \item \texttt{initial\_elevation}: Unghiul de elevație (în grade) al camerei virtuale pentru vizualizarea 3D.
    \item \texttt{output\_filename\_gif}: Numele fișierului GIF în care va fi salvată animația.
\end{itemize}

\subsection{Configurarea Scenei 3D}
O figură și un subplot 3D sunt create folosind `matplotlib`. Fundalul figurii și al zonei de desenare a axelor este setat explicit pe alb.
\begin{lstlisting}[language=Python, firstnumber=16, caption=Configurarea figurii și axelor]
fig = plt.figure(figsize=(9, 8))
ax = fig.add_subplot(111, projection='3d')
fig.patch.set_facecolor('white') 
ax.set_facecolor('white') 
\end{lstlisting}

\subsection{Generarea Liniilor Riglate}
Se definesc vectorii \texttt{s\_values} (coordonatele $z$ de-a lungul unei drepte) și \texttt{alpha\_values} (parametrii unghiulari pentru distribuirea dreptelor în jurul axei $z$).
Apoi, pentru fiecare familie de linii:
\begin{itemize}
    \item Se iterează prin valorile \texttt{alpha}.
    \item Pentru fiecare \texttt{alpha}, se calculează coordonatele $(x, y, z)$ ale punctelor de pe acea dreaptă folosind ecuațiile parametrice menționate anterior.
    \item Datele fiecărei linii (coordonate, culoare, grosime) sunt stocate într-o listă \texttt{lines\_data}. Familia 1 este colorată în albastru, iar familia 2 în roșu.
\end{itemize}
\begin{lstlisting}[language=Python, firstnumber=26, caption=Generarea datelor pentru o familie de linii]
# familia 1
for alpha in alpha_values:
    x_line1 = a * (np.cos(alpha) - (s_values / c_eq) * np.sin(alpha))
    y_line1 = b * (np.sin(alpha) + (s_values / c_eq) * np.cos(alpha))
    z_line1 = s_values
    lines_data.append({'x': x_line1, 'y': y_line1, 'z': z_line1, 'color': 'blue', 'linewidth': 1.2})
\end{lstlisting}

\subsection{Funcția de Animație \texttt{update\_frame}}
Această funcție este esențială și este apelată pentru fiecare cadru al animației:
\begin{enumerate}
    \item \texttt{ax.clear()}: Șterge conținutul axelor din cadrul anterior.
    \item \texttt{ax.set\_facecolor('white')}: Reaplică fundalul alb al axelor, deoarece \texttt{clear()} poate reseta unele proprietăți.
    \item \textbf{Redesenarea Liniilor}: Iterează prin \texttt{lines\_data} și desenează fiecare linie cu proprietățile stocate.
    \item \textbf{Configurarea Axelor}: 
        \begin{itemize}
            \item \texttt{ax.set\_xticks([]), ax.set\_yticks([]), ax.set\_zticks([])}: Elimină gradațiile numerice de pe axe.
            \item \texttt{ax.set\_xlabel(""), ax.set\_ylabel(""), ax.set\_zlabel("")}: Elimină etichetele axelor ("X-axis", etc.).
            \item \texttt{ax.set\_title("Cuadrică dublu-riglată")}: Setează un titlu pentru grafic.
        \end{itemize}
    \item \textbf{Setarea Limitelor Axelor}: Limitele axelor $(x, y, z)$ sunt calculate și setate explicit. Acest pas este crucial pentru a menține o vizualizare consistentă și a preveni redimensionarea automată a scenei între cadre, ceea ce ar putea duce la o animație "săltăreață".
    \item \textbf{Rotirea Vederii}: \texttt{ax.view\_init(elev=initial\_elevation, azim=azimuth\_angle\_deg)} actualizează unghiul de azimut al camerei, creând efectul de rotație a obiectului în jurul axei verticale.
\end{enumerate}

\subsection{Crearea și Salvarea Animației}
\begin{itemize}
    \item \texttt{FuncAnimation}: Această clasă din Matplotlib este utilizată pentru a crea obiectul de animație, apelând repetitiv funcția \texttt{update\_frame} pentru fiecare cadru. Intervalul dintre cadre este determinat de \texttt{fps}.
    \item \texttt{PillowWriter}: Animația este salvată într-un fișier GIF folosind \texttt{PillowWriter} din biblioteca Pillow (o dependență a Matplotlib pentru scrierea GIF-urilor).
\end{itemize}

\section{Codul Sursă Python Complet}
\begin{lstlisting}[language=Python, caption=Script Python pentru animația cuadricei dublu riglate, label=lst:python_code]
import numpy as np
import matplotlib.pyplot as plt
from mpl_toolkits.mplot3d import Axes3D
from matplotlib.animation import FuncAnimation, PillowWriter

# --- parametrii hiperboloidului ---
a = 2.0          # semiaxa asociata x pentru elipsa din z=0
b = 1.5          # semiaxa asociata y pentru elipsa din z=0
c_eq = 2.0       # parametrul 'c' din x^2/a^2 + y^2/b^2 - z^2/c_eq^2 = 1
                 # Smaller c_eq -> more pronounced waist/skew.
                 # micsorarea parametrului c_eq modifica inclinatia dreptelor

z_max = 3.0      # lungimea unei drepte
num_lines = 20   # numarul de drepte dintr-o familie
num_points_per_line = 50 # arbitrar, numarul de puncte de pe o dreapta

# --- parametrii animatiei ---
num_frames = 360         # numarul total de frame-uri pentru o rotatie completa
fps = 60                 # cadre pe secunda
initial_elevation = 25   # unghiul de elevatie
output_filename_gif = "animatie_cuadrica_dubluriglata.gif"

# --- setup axe 3D ---
fig = plt.figure(figsize=(9, 8)) # ajustare marime
ax = fig.add_subplot(111, projection='3d')
fig.patch.set_facecolor('white') # fundal alb
ax.set_facecolor('white') # axe de reprezentare albe

# --- generarea datelor despre linii ---
s_values = np.linspace(-z_max, z_max, num_points_per_line)
alpha_values = np.linspace(0, 2 * np.pi, num_lines, endpoint=False)

lines_data = []

# familia 1
for alpha in alpha_values:
    x_line1 = a * (np.cos(alpha) - (s_values / c_eq) * np.sin(alpha))
    y_line1 = b * (np.sin(alpha) + (s_values / c_eq) * np.cos(alpha))
    z_line1 = s_values
    lines_data.append({'x': x_line1, 'y': y_line1, 'z': z_line1, 'color': 'blue', 'linewidth': 1.2})

# familia 2
for alpha in alpha_values:
    x_line2 = a * (np.cos(alpha) + (s_values / c_eq) * np.sin(alpha))
    y_line2 = b * (np.sin(alpha) - (s_values / c_eq) * np.cos(alpha))
    z_line2 = s_values
    lines_data.append({'x': x_line2, 'y': y_line2, 'z': z_line2, 'color': 'red', 'linewidth': 1.2})

# --- functia de animare ---

def update_frame(frame_number): # functia de desenare a unui singur cadru
    ax.clear() # reimprospatarea cadrului
    ax.set_facecolor('white') # reaplicarea fundalului alb

    for line_props in lines_data: # improspatarea pozitiilor liniilor
        ax.plot(line_props['x'], line_props['y'], line_props['z'],
                color=line_props['color'], linewidth=line_props['linewidth'])

    ax.set_xticks([])
    ax.set_yticks([])
    ax.set_zticks([])
    ax.set_xlabel("")
    ax.set_ylabel("")
    ax.set_zlabel("")
    ax.set_title("")
    ax.grid(False)
    ax.set_title("Cuadrica dublu-riglata")

    # extinderea maxima a liniilor pana la  s = z_max
    # pentru x: a * (cos(alpha) - (z_max/c_eq) * sin(alpha)). valoarea maxima este: a * sqrt(1 + (z_max/c_eq)^2)
    max_coord_factor = np.sqrt(1 + (z_max/c_eq)**2)
    axis_limit_x = a * max_coord_factor * 1.1
    axis_limit_y = b * max_coord_factor * 1.1
    axis_limit_z = z_max * 1.1

    ax.set_xlim([-axis_limit_x, axis_limit_x])
    ax.set_ylim([-axis_limit_y, axis_limit_y])
    ax.set_zlim([-axis_limit_z, axis_limit_z])
    ax.grid(True) # This will make the grid visible

    azimuth_angle_deg = frame_number * (360 / num_frames)
    ax.view_init(elev=initial_elevation, azim=azimuth_angle_deg)

    return fig,

print("generarea cadrelor animatiei")
animation = FuncAnimation(fig, update_frame, frames=num_frames,
                          interval=(1000 / fps), blit=False)

try:
    print(f"salvarea animatiei ca: {output_filename_gif}")
    writer = PillowWriter(fps=fps)
    animation.save(output_filename_gif, writer=writer)
    print(f"animatie salvata ca: {output_filename_gif}")
except Exception as e:
    print(f"eroare salvare: {e}")

print("script terminat")
\end{lstlisting}

\section{Concluzii}
Scriptul Python prezentat generează o animație a unui hiperboloid cu o pânză, ilustrând clar cele două familii de generatoare care formează suprafața sa. Prin ajustarea parametrilor, se pot explora diverse forme și configurații ale acestei cuadrice dublu riglate.

\end{document}